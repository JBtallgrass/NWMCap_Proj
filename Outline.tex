
---

Capstone Report Title (Tentative)
Bridging the Gap: Aligning Education and Employment in North Central Arkansas through Data-Driven Workforce Development

1. Executive Summary

Brief overview of the project purpose, methods, findings, and key recommendations
Focus on impact: education programs vs labor demand, regional economic health, and data insights.

2. Problem Statement & Objectives

Define the regional challenges: rurality, economic constraints, and the digital divide.
Central question: Are educational pathways aligned with labor market needs in North Central Arkansas?
Objectives:
\begin{itemize}
    \item Identify current education/employment mismatches
    \item Predict future workforce needs
    \item Provide actionable recommendations
\end{itemize}

3. Background & Regional Context
\begin{itemize}
    \item Demographic and economic overview of North Central Arkansas
    \item Education system snapshot (K-12, technical, college)
    \item Labor force statistics and top industries
    \item Broadband and technology infrastructure considerations
\end{itemize}

4. Methodology (CRISP-DM Framework)
\begin{itemize}
    \item Business Understanding
    \item Data Understanding
    \item Data Preparation
    \item Modeling
    \item Evaluation
    \item Deployment (focused on policy & practice)
\end{itemize}
*Link to supplemental `CRISP-DM.md` with detailed documentation for each phase*

---

## 5. **Data Sources & Collection**

* Overview of datasets used (IPEDS, ACS, BLS, O\*NET, Arkansas Dept. of Education, local institutions)
* Data limitations, quality, and preprocessing decisions
* Timeline and region filtering (North Central Arkansas counties)

---

## 6. **Exploratory Data Analysis (EDA)**

* Key visualizations: population trends, education attainment, job postings, skills gap
* Insights on under- or oversupplied career fields
* Socioeconomic factors (transportation, internet access)

---

## 7. **Predictive Modeling**

* Overview of models (e.g., Linear Regression for enrollment forecasting, Classification for job alignment)
* Future workforce trends (5-year outlook using historical data)
* Scenario planning: what if broadband access improves?

---

## 8. **Findings & Interpretation**

* Skill mismatches and education pipeline gaps
* Regional disparities (urban vs rural)
* High-opportunity sectors with low program representation
* Community college and adult education alignment

---

## 9. **Recommendations**

* Education: create or expand high-demand short-cycle programs
* Policy: invest in broadband, transportation, and data-sharing ecosystems
* Outreach: employer-education partnerships, career awareness for youth
* Further research: how seasonal tourism or remote work affects trends

---

## 10. **Conclusion & Next Steps**

* Summary of actionable insights
* How this project can inform workforce boards, colleges, or chambers
* Recommendations for continued data tracking and community involvement

---

## 11. **Appendices**

* County-level data summaries
* Methodological notes
* Supplemental visuals or model outputs
* Glossary of terms

---

## 12. **References**

* APA citations for all data sources and background research

---

Would you like this adapted for PDF or PowerPoint presentation as well?
