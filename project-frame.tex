CRISP-DM Project Structure  
Workforce Readiness in Rural North Central Arkansas

This document provides a detailed breakdown of how the CRISP-DM (Cross-Industry Standard Process for Data Mining) methodology is applied to this project, which explores the alignment between educational outcomes and labor market demands in rural North Central Arkansas.

1. Business Understanding

Objective
To investigate whether current educational outcomes—such as certifications and associate degrees—are aligned with labor market needs in rural counties of North Central Arkansas.

Key Problem
Rural counties often experience low educational attainment, limited broadband access, and a mismatch between training programs and available jobs. This project aims to identify patterns in workforce readiness and provide actionable insights for regional planning and educational strategy.

Target Region
Counties include (but are not limited to):
- Baxter  
- Fulton  
- Marion  
- Izard  
- Sharp  
- Stone  

Stakeholders
- Arkansas State University–Mountain Home (ASUMH)
- Local workforce boards
- State education policymakers
- Rural economic development partners

2. Data Understanding

Primary Data Sources
| Source | Description |
|--------|-------------|
| ACS (Census) | Demographics, education, broadband, poverty |
| BLS | Employment stats, job openings, wages |
| USDA ERS | Rural typologies, persistent poverty indicators |
| IPEDS | Postsecondary completions, enrollment |
| Arkansas Open Data Portal | Local training data, economic indicators |

Exploratory Steps
- Reviewed variable distributions across counties
- Identified missing values, scaling issues, and categorical encodings
- Cross-referenced multiple sources to validate educational and job data

3. Data Preparation

Processing
- Cleaned raw datasets for consistency (e.g., column names, null handling)
- Merged datasets using county-level FIPS codes
- Filtered data to only include relevant counties and years
- Removed or imputed missing values where needed

Feature Engineering
- `% adults with associate degrees or higher`
- `Job postings per capita in high-demand fields`
- `Broadband subscription %`
- `Unemployment rate`
- `IPEDS completions in technical or healthcare programs`

Output
Final structured dataset:  
`data/final/nca_workforce_ready.csv`

Modeling

Unsupervised Learning – Clustering
Goal: Segment counties into clusters based on shared characteristics

Techniques Used:
- K-Means Clustering
- Hierarchical Agglomerative Clustering
- DBSCAN (exploratory)

Features Used for Clustering:
- Educational attainment
- Broadband access
- Job demand in middle-skill sectors
- Postsecondary completions

Evaluation Metrics:
- Silhouette Score
- Davies-Bouldin Index
- Dendrogram interpretation

Supervised Learning (Optional)
Goal: Predict counties most at risk of workforce development gaps

Techniques Considered:
- Logistic Regression
- Random Forest Classifier
- Ridge Regression (if modeling numeric risk scores)

Target Variable (if modeled):
- Binary or categorical risk tier based on education-to-employment misalignment

5. Evaluation

Clustering Review
- Visualized clusters with PCA, scatter plots, and county-level maps
- Compared internal validation scores (silhouette, inertia)
- Interpreted cluster profiles to determine real-world meaning:
  - Cluster A: High broadband, low completions  
  - Cluster B: Low education, high poverty  
  - Cluster C: Workforce aligned

Supervised Model (if used)
- Analyzed feature importance and confusion matrix
- Assessed risk prediction validity using cross-validation
- Discussed ethical limitations in labeling "at-risk" counties

6. Deployment & Reporting

Visual Deliverables
- County Cluster Map (choropleth or bubble map)
- Cluster Comparison Charts (bar, slope, dot)
- Feature Relationships (scatterplots, regression line comparisons)
- (Optional) Tableau dashboard for exploration

Final Report Includes:
- Executive summary for stakeholders
- Description of method and findings
- Visuals and interpretations
- Policy or curriculum recommendations

Deliverables
- `/report/Executive_Summary.pdf`
- `/visuals/cluster_map.png`
- `/notebooks/final_model_analysis.ipynb`

Final Notes

This CRISP-DM structure enabled a methodical and adaptable approach to addressing a real-world workforce question in a rural context. While model performance and data availability are constraints, this framework facilitated exploration, insight generation, and evidence-based recommendations that can support rural resilience through education.


Prepared by Jason  
Data Scientist and Rural Advocate  
June 2025
